\documentclass{informatics}
%%%%%%%%%%%%%%%%%%%%%%%%%%%%%
%%%%%% Question scores %%%%%%
%%%%%%%%%%%%%%%%%%%%%%%%%%%%%
\def\truefalseScore{v=0,b=1,m=-1}
\def\singleScore{e=0,v=0,b=1,m=0}
\def\multipleScore{e=0,v=0,b=.25,m=-.25,p=0,d=0}

%%%%%%%%%%%%%%%%%%%%%%%%%%%%%
%%%%%%%%% Score box %%%%%%%%%
%%%%%% for open answer %%%%%% 
%%%%%%%%% questions %%%%%%%%% 
%%%%%%%%%%%%%%%%%%%%%%%%%%%%%
\newcommand{\scorebox}{
  \wrongchoice[0]{0}\scoring{0}
  \wrongchoice[.1]{1}\scoring{0.1}
  \wrongchoice[.2]{2}\scoring{0.2}
  \wrongchoice[.3]{3}\scoring{0.3}
  \wrongchoice[.4]{4}\scoring{0.4}
  \wrongchoice[.5]{5}\scoring{0.5}
  \wrongchoice[.6]{6}\scoring{0.6}
  \wrongchoice[.7]{7}\scoring{0.7}
  \wrongchoice[.8]{8}\scoring{0.8}
  \wrongchoice[.9]{9}\scoring{0.9}
  \correctchoice[1]{10}\scoring{1}
}

%%%%%%%%%%%%%%%%%%%%%%%%%%%%%
%% Mutiple question symbol %%
%%%% Refer to AMC's docs %%%%
%%%%%%%%%%%%%%%%%%%%%%%%%%%%%
% \def\multiSymbole{$\clubsuit$}

%%%%%%%%%%%%%%%%%%%%%%%%%%%%%
%%%%%%%%% Margins %%%%%%%%%%%
%%%%%%%%%%%%%%%%%%%%%%%%%%%%%
\geometry{hmargin=1.5cm,headheight=1cm,headsep=.3cm,footskip=1cm,top=2.5cm,bottom=2cm}


\begin{document}

%%%%%%%%%%%%%%%%%%%%%%%%%%%%%%
%%% Add your question banks %%
%%%%%%%%%%%%%%%%%%%%%%%%%%%%%%
% \begin{pool}{example-pool}
  \truequestion{This is a true question}
  \falsequestion{This is a false question}
  \truequestion{This is another true question}
  \falsequestion{This is another false question}
\end{pool}


%%%%%%%%%%%%%%%%%%%%%%%%%%%%%%
%%% Set the number of exams %%
%%%%%%% One per student %%%%%%
%%%%%%%%%%%%%%%%%%%%%%%%%%%%%%
\begin{examcopy}[5]

  \AMCformBegin

  %%%%%%%%%%%%%%%%%%%%%%%%%%%%%%
  %%%%% Set the exam header %%%%
  %%%%%%%%%%%%%%%%%%%%%%%%%%%%%%
  \header{<Discipline>}
    {<Exam Type>}
    {Engenharia Informática}
    {<Date>}
    {<Duration>}
    {20 Valores}
    {Preencha o nome e o número de aluno no topo desta folha. Adicionalmente,
    \underline{pinte completamente} ($\blacksquare$) as caixas
    correspondentes aos dígitos do número na grelha ao lado (\underline{dígito das
    unidades na caixa mais à direita}). Proceda da mesma forma para assinalar
  as respostas que considera correctas. 
  \\ \textbf{Não use as áreas sombreadas!}}

  \AMCform

  % Reset counters for each exam copy
  % DO NOT CHANGE 
  \setcounter{groupCounter}{0}
  
  %%%%%%%%%%%%%%%%%%%%%%%%%%%%%%
  %%%%%%%%% Begin exam  %%%%%%%%
  %%%%%%%%%%%%%%%%%%%%%%%%%%%%%%

  % Build your exam here
  % \insertpool{example-pool}

  %%%%%%%%%%%%%%%%%%%%%%%%%%%%%%
  %%%%%%%%%% End exam  %%%%%%%%%
  %%%%%%%%%%%%%%%%%%%%%%%%%%%%%%

\end{examcopy}
\end{document}
